
\documentclass[letter, fleqn, 11pt]{article}
%%%%%%%%%%%%%%%%%%%%%%%%%%%%%%%%%%%%%%%%%%%%%%%%%%%%%%%%%%%%%%%%%%%%%%%%%%%%%%%%%%%%%%%%%%%%%%%%%%%%%%%%%%%%%%%%%%%%%%%%%%%%%%%%%%%%%%%%%%%%%%%%%%%%%%%%%%%%%%%%%%%%%%%%%%%%%%%%%%%%%%%%%%%%%%%%%%%%%%%%%%%%%%%%%%%%%%%%%%%%%%%%%%%%%%%%%%%%%%%%%%%%%%%%%%%%
\usepackage{makeidx}
\usepackage{amsfonts}
\usepackage[authoryear]{natbib}
\usepackage{amsmath}
\usepackage{amsthm}
\usepackage{graphicx}
\usepackage[authoryear]{natbib}
\usepackage{hyperref}
\usepackage{showlabels}
\usepackage{mathptmx}
\usepackage{color}
\usepackage[sf,bf,center]{titlesec}
\usepackage{fancyhdr}
\usepackage{setspace}

\setcounter{MaxMatrixCols}{10}
%TCIDATA{OutputFilter=LATEX.DLL}
%TCIDATA{Version=5.00.0.2552}
%TCIDATA{<META NAME="SaveForMode" CONTENT="1">}
%TCIDATA{LastRevised=Thursday, September 04, 2008 22:42:16}
%TCIDATA{<META NAME="GraphicsSave" CONTENT="32">}

\newtheorem{theorem}{Theorem}
\newtheorem{acknowledgement}[theorem]{Acknowledgement}
\newtheorem{algorithm}[theorem]{Algorithm}
\newtheorem{axiom}[theorem]{Axiom}
\newtheorem{case}[theorem]{Case}
\newtheorem{claim}[theorem]{Claim}
\newtheorem{conclusion}[theorem]{Conclusion}
\newtheorem{condition}[theorem]{Condition}
\newtheorem{conjecture}[theorem]{Conjecture}
\newtheorem{corollary}[theorem]{Corollary}
\newtheorem{criterion}[theorem]{Criterion}
\newtheorem{definition}[theorem]{Definition}
\newtheorem{example}[theorem]{Example}
\newtheorem{exercise}[theorem]{Exercise}
\newtheorem{lemma}[theorem]{Lemma}
\newtheorem{notation}[theorem]{Notation}
\newtheorem{problem}[theorem]{Problem}
\newtheorem{proposition}[theorem]{Proposition}
\newtheorem{remark}[theorem]{Remark}
\newtheorem{solution}[theorem]{Solution}
\newtheorem{summary}[theorem]{Summary}
\newtheorem{assumption}[theorem]{Assumption}
\makeatletter \chead{Learning under Fear}
\newlength{\myFootnoteWidth}
\newlength{\myFootnoteLabel}
\setlength{\myFootnoteLabel}{.975em}
\renewcommand{\@makefntext}[1]{  \setlength{\myFootnoteWidth}{\columnwidth}  \addtolength{\myFootnoteWidth}{-\myFootnoteLabel}  \noindent\makebox[\myFootnoteLabel][r]{\@makefnmark\ }  \parbox[t]{\myFootnoteWidth}{#1}} \makeatother
\setlength{\parindent}{0in} \setlength{\parskip}{6pt} \addtolength{\hoffset}{-1.cm} \addtolength{\textwidth}{2cm} \addtolength{\voffset}{-1.cm}
\addtolength{\textheight}{2cm} \frenchspacing \pagenumbering{arabic} \pretolerance=2000 \tolerance=3000
\hypersetup{colorlinks=true,citecolor=blue, filecolor = red, linkcolor = red,breaklinks=true,frenchlinks=true} \onehalfspacing
%\input{tcilatex}

\begin{document}

\title{Running LQ Robust Regulator Problems in Dynare}
\date{February 2010 - Preliminary Version}
\author{Prepared by the Artificial "Evil Agent"}
\maketitle

\section{Goal of this Document}
This document shows how to formulate Linear Quadratic (LQ) problems that feature a decision maker that fears he or she might use a misspecified model of the economy.
We borrow the LQ environment from Hansen \& Sargent (Recursive Models of Dynamic Linear Economies) and present matlab files that take as input the matrices governing the economic environment in Hansen \& Sargent and use the Dynare package to compute equilibrium outcomes and possibly estimate model parameters.
Finally, we illustrate the matlab code using as an example the economy described by Hansen, Sargent \& Tallarini (1999).
%This document is to explain how the Linear Quadratic LQ problem setup of Hansen and Sargent (HS) can be easily adapted for LQ
%Robustness problems and solved using dynare. We present some examples. We use a matlab subroutine that does the job of building a dynare
%\texttt{.mod} file to write a model that one can simulate and potentially estimate by inputting the matrixes in HS setup.

\section{Hansen Sargent LQ Setup}

This section describes the standard LQ setup that we borrow from Hansen \& Sargent.

\begin{equation*}
\max_{\left\{ c_{t},k_{t}\right\} }-\left( 1/2\right) \sum_{t=0}^{\infty
}\beta ^{t}\left[ (s_{t}-b_{t})\cdot (s_{t}-b_{t})+(g_{t})\cdot (g_{t})%
\right]
\end{equation*}%
subject to the following set of restrictions:

\begin{eqnarray*}
\Phi _{c}c_{t}+\Phi _{g}g_{t}+\Phi _{i}i_{t} &=&\Gamma k_{t-1}+d_{t} \\
k_{t} &=&\Delta _{k}k_{t-1}+\Theta _{k}i_{t} \\
h_{t} &=&\Delta _{h}h_{t-1}+\Theta _{h}c_{t} \\
s_{t} &=&\Lambda h_{t-1}+\Pi c_{t}
\end{eqnarray*}

The exact interpretation of these equations will depend on the economic problem studied, but roughly speaking, the terms involved are: $s_{t}$ is a variable
that maps consumption units today, $c_{t}$, and a stock of habits, $h_{t},$ into utilities, while $b_{t}$ is a shock affecting utility. $g_{t}$
scales the utility, and may be seen as lump-sum transfers. The first restriction is a resource constraint. The second is the evolution of stocks
of capital and the third is the process for the formation of habits (but these may also be seen as the accumulation of consumer durable after a
purchase $c_{t}$)$.$

Additionally, the random variables evolve according to:

\begin{equation*}
z_{t+1}=A_{22}z_{t}+C_{2}w_{t+1},b_{t}=U_{b}z_{t}\text{ and }d_{t}=U_{d}z_{t}
\end{equation*}%
$w_{t+1}$ is Gaussian vector with identity covariance matrix.

The agent that is maximizing these equations chooses a sequence $%
c_{t,}i_{t},k_{t},h_{t},g_{t}$ contingent on the state $%
k_{t-1},h_{t-1},z_{t} $. Given the convexity of the objective function, we can setup the lagrangian and it will yield a set of first order
necessary conditions (FONCs). The lagrangian will therefore be:

\begin{eqnarray*}
&&-\left( 1/2\right) E\{\sum_{t=0}^{\infty }\beta ^{t}\left[
(s_{t}-b_{t})\cdot (s_{t}-b_{t})+(g_{t})\cdot (g_{t})\right] +... \\
&&+{\scriptsize M}_{t}^{d\prime }\left[ \Phi _{c}c_{t}+\Phi _{g}g_{t}+\Phi
_{i}i_{t}-\Gamma k_{t-1}-d_{t}\right] ... \\
&&+{\scriptsize M}_{t}^{k\prime }\left[ k_{t}-\Delta _{k}k_{t-1}-\Theta
_{k}i_{t}\right] ... \\
&&+{\scriptsize M}_{t}^{h\prime }\left[ h_{t}-\Delta _{h}h_{t-1}-\Theta
_{h}c_{t}\right] ... \\
&&+{\scriptsize M}_{t}^{s\prime }\left[ s_{t}-\Lambda h_{t-1}-\Pi c_{t}%
\right] \}
\end{eqnarray*}

%This lagrangian corresponds to HS's equation...

\section{First Order Conditions}

The FONCs for the problem, found in system 4.2.4 of Hansen and Sargent are:

\begin{eqnarray*}
\left( c_{t}\right)  &:&-{\scriptsize M}_{t}^{d\prime }\Phi _{c}+%
{\scriptsize M}_{t}^{h\prime }\Theta _{h}+{\scriptsize M}_{t}^{s\prime }\Pi
=0 \\
\left( g_{t}\right)  &:&g_{t}-\Phi _{g}{\scriptsize M}_{t}^{d}=0 \\
\left( h_{t}\right)  &:&-{\scriptsize M}_{t}^{h\prime }+\beta E\left[
{\scriptsize M}_{t+1}^{h\prime }\Delta _{h}+{\scriptsize M}_{t+1}^{s\prime
}\Lambda \right] =0 \\
\left( i_{t}\right)  &:&-{\scriptsize M}_{t}^{d\prime }\Phi _{i}+%
{\scriptsize M}_{t}^{k\prime }\Theta _{i}=0 \\
\left( k_{t}\right)  &:&-{\scriptsize M}_{t}^{k\prime }\Delta _{k}+\beta E%
\left[ {\scriptsize M}_{t+1}^{d\prime }\Gamma +{\scriptsize M}%
_{t+1}^{k\prime }\Delta _{k}\right] =0 \\
\left( s_{t}\right)  &:&-(s_{t}-b_{t})+{\scriptsize M}_{t}^{s}=0
\end{eqnarray*}

and it represents a system of linear difference equations of order n, where the order depends on the nature of the problem studied. The reason
why the derivative of the squared terms does not have a 2 in front is because we premultiplied the objective by 1/2. You may also note that the
second and sixth blocks have the matrixes transposed. This is a result of taking the derivatives of a variable that appears in the objective
function, which yields the vector itself, and in the restriction which yields the j-th variable, the sum of the j-th row of whatever matrix they
multiply and the j-th multiplier. This equations together with the restrictions determine the solution:

\begin{eqnarray*}
\Phi _{c}c_{t}+\Phi _{g}g_{t}+\Phi _{i}i_{t} &=&\Gamma k_{t-1}+d_{t} \\
k_{t} &=&\Delta _{k}k_{t-1}+\Theta _{k}i_{t} \\
h_{t} &=&\Delta _{h}h_{t-1}+\Theta _{h}c_{t} \\
s_{t} &=&\Lambda h_{t-1}+\Pi c_{t} \\
b_{t} &=&U_{b}z_{t} \\
d_{t} &=&U_{d}z_{t}
\end{eqnarray*}

\section{LQ Robustness Setup}

The problem of a decision maker that fears model misspecification for the same economic problem has a similar
construction. It is described by a similar setup in which to agents play the
following game:%
\begin{equation*}
\min_{\left\{ w_{t+1}\right\} }\max_{\left\{ c_{t},k_{t}\right\}
}-E\{\sum_{t=0}^{\infty }\beta ^{t}\left( \frac{1}{2}\right) \left[
(s_{t}-b_{t})\cdot (s_{t}-b_{t})+(g_{t})\cdot (g_{t})\right] -\beta \theta
w_{t+1}\cdot w_{t+1}\}
\end{equation*}%
subject to the following set of restrictions:

\begin{eqnarray*}
\Phi _{c}c_{t}+\Phi _{g}g_{t}+\Phi _{i}i_{t} &=&\Gamma k_{t-1}+d_{t} \\
k_{t} &=&\Delta _{k}k_{t-1}+\Theta _{k}i_{t} \\
h_{t} &=&\Delta _{h}h_{t-1}+\Theta _{h}c_{t} \\
s_{t} &=&\Lambda h_{t-1}+\Pi c_{t} \\
b_{t} &=&U_{b}z_{t} \\
d_{t} &=&U_{d}z_{t}
\end{eqnarray*}%
and the minimizer or "evil" agent chooses $w_{t+1},$ that will affect:

\begin{equation*}
z_{t+1}=A_{22}z_{t}+C_{2}w_{t+1}
\end{equation*}%
$w_{t+1}$ is not an exogenous Gaussian vector any more, but corresponds to a chosen process by the evil agent (It will happen though that in these types of
problems, the worst choice happens to be Gaussian too with mean $w_{t+1}$! and this is a feature we exploit later).
Then, these shocks are transformed accordingly to $%
b_{t}=U_{b}z_{t}$ and $d_{t}=U_{d}z_{t}.$

The Robustness problem can also be written as a Lagrangian:

\begin{eqnarray*}
&&\min_{\left\{ w_{t+1}\right\} }\max_{\left\{ c_{t},k_{t}\right\}
}-\{\sum_{t=0}^{\infty }\left( 1/2\right) \beta ^{t}\left[
(s_{t}-b_{t})\cdot (s_{t}-b_{t})+(g_{t})\cdot (g_{t})\right] -\beta \theta
\frac{w_{t+1}\cdot w_{t+1}}{2}\} \\
&&+{\scriptsize M}_{t}^{d\prime }\left[ \Phi _{c}c_{t}+\Phi _{g}g_{t}+\Phi
_{i}i_{t}-\Gamma k_{t-1}-d_{t}\right]  \\
&&+{\scriptsize M}_{t}^{k\prime }\left[ k_{t}-\Delta _{k}k_{t-1}-\Theta
_{k}i_{t}\right]  \\
&&+{\scriptsize M}_{t}^{h\prime }\left[ h_{t}-\Delta _{h}h_{t-1}-\Theta
_{h}c_{t}\right]  \\
&&+{\scriptsize M}_{t}^{s\prime }\left[ s_{t}-\Lambda h_{t-1}-\Pi c_{t}%
\right]  \\
&&+{\scriptsize M}_{t}^{z\prime }\left[ -z_{t+1}+A_{22}z_{t}+C_{2}w_{t+1}%
\right] \}
\end{eqnarray*}

\section{The Robust\ Solution}

The set of linear equations determining the outcome will be similar to those of the
original LQ problem:

\begin{eqnarray}
\left( c_{t}\right)  &:&-{\scriptsize M}_{t}^{d\prime }\Phi _{c}+%
{\scriptsize M}_{t}^{h\prime }\Theta _{h}+{\scriptsize M}_{t}^{s\prime }\Pi
=0  \label{SYSTEM1} \\
\left( g_{t}\right)  &:&g_{t}-\Phi _{g}{\scriptsize M}_{t}^{d}=0  \notag \\
\left( h_{t}\right)  &:&-{\scriptsize M}_{t}^{h\prime }+\beta E\left[
{\scriptsize M}_{t+1}^{h\prime }\Delta _{h}+{\scriptsize M}_{t+1}^{s\prime
}\Lambda \right] =0  \notag \\
\left( i_{t}\right)  &:&-{\scriptsize M}_{t}^{d\prime }\Phi _{i}+%
{\scriptsize M}_{t}^{k\prime }\Theta _{i}=0  \notag \\
\left( k_{t}\right)  &:&-{\scriptsize M}_{t}^{k\prime }\Delta _{k}+\beta E%
\left[ {\scriptsize M}_{t+1}^{d\prime }\Gamma +{\scriptsize M}%
_{t+1}^{k\prime }\Delta _{k}\right] =0  \notag \\
\left( s_{t}\right)  &:&-(s_{t}-b_{t})+{\scriptsize M}_{t}^{s}=0  \notag \\
\left( w_{t+1}\right)  &:&\beta \theta w_{t+1}+C_{2}^{^{\prime }}%
{\scriptsize M}_{t}^{z}=0  \notag \\
\left( z_{t+1}\right)  &:&-\beta E\left[ U_{d}^{\prime }{\scriptsize M}%
_{t+1}^{d}\right] +\beta E\left[ A_{22}^{\prime }{\scriptsize M}_{t+1}^{z}%
\right] -\beta E\left[ \left( s_{t+1}-b_{t+1}\right) U_{b}^{\prime }%
{\scriptsize M}_{t+1}^{z}\right] -{\scriptsize M}_{t}^{z}=0  \notag
\end{eqnarray}

This set of equations is then complemented by:

\begin{eqnarray*}
\Phi _{c}c_{t}+\Phi _{g}g_{t}+\Phi _{i}i_{t} &=&\Gamma k_{t-1}+d_{t} \\
k_{t} &=&\Delta _{k}k_{t-1}+\Theta _{k}i_{t} \\
h_{t} &=&\Delta _{h}h_{t-1}+\Theta _{h}c_{t} \\
s_{t} &=&\Lambda h_{t-1}+\Pi c_{t}
\end{eqnarray*}%
and the minimizer or "evil" agent chooses $w_{t+1},$ that will affect:

\begin{equation*}
z_{t+1}=A_{22}z_{t}+C_{2}\left( w_{t+1}+\epsilon _{t}\right)
\end{equation*}%
where $\epsilon _{t}\sim N\left( 0,I\right) .$

Before explaining how the matlab code works we need to explain how we map the worst case distortions into our LQ framework. Notice that $\epsilon _{t}$ was not appearing in our original
problem. In the original Robustness Setup, the evil agent chooses a distribution over the sequences of outcomes $w_{t},$ but not the outcomes
themselves. %Here, $\epsilon _{t}$ is a randomization device.
By the certainty equivalence principle that shows up in LQ problems, we know that
for any choice of the distribution (in the standard setup) for $w_{t}$, the maximizer's optimal behavior will depend only on the first moment.
Hence the choice of $w_{t+1}$ in this setup is equivalent to the choice of the mean of a particular
process. But why is then $\epsilon _{t}$ also Gaussian, and moreover $%
N\left( 0,I\right) ?$ Well, the distorted probability will behave like we
changed the maximizer agents model's mean zero Gaussian shocks to $N\left(
w_{t+1},I\right) $. Gaussianity is a property of the Robust Problem when
the approximating model is Gaussian. The i.i.d with variance 1 of the shocks
is in fact, not strictly correct: the worst Gaussian Variance-Covariance
matrix in fact needs to be computed differently, but at this stage, we only
care for the optimal policy of the maximizing agent and the mean of the
worst case scenario and this is sufficient. Please take this into account when interpreting (and scaling) impulse responses. %Having said the we proceed to
%explain how the codes work.
We also include in our zip file of matlab codes a file called \begin{verbatim} HSLQ_dyn.m \end{verbatim} that computes the standard LQ problem without robustness described above. That file treats $w$ directly as a vector of exogenous disturbances. 
\newpage

\section{Using Dynare to Compute Robust Policies}

The \texttt{HS\_Robust\_dyn}.m code enables the user to compute robust policies with dynare by simply by inputting matrixes from the LQ setup.
The program reads matrixes from the LQ setup and uses the general solution to the problem to write a MOD file for the system of stochastic
difference equations. The instructions are very simple.

\subsection{What to do?}

\textbf{What to do?} The user must write up a simple m-file to use to declare all of the matrixes in the linear quadratic setup and declare some
of the optional parameters. The user should write up an m-file with that includes the following:

\begin{enumerate}
\item \textbf{Declaration of matrixes:} The first thing the user has to do
is declare ALL of these matrixes: \texttt{Phi\_c, Phi\_g, Phi\_i, Gamma,
Delta\_k, Theta\_k, Delta\_h,.Theta\_h, Lambda, Pi, A22,C2, U\_b, U\_d};

\item \textbf{Declaration of options for Dynare: }Variables that go in the standard set of options for dynare should be included. For example
one can include statements like:

\begin{itemize}
\item \texttt{periods=5000;} (for number of simulations in Dynare)

\item \texttt{irfperiods=15; }(for number of periods in irf's in Dynare)
\end{itemize}

\item \textbf{File Name:} A statement assigning the variable \texttt{filename%
} a script should be included:(e.g. \texttt{filename ='Hall\_Test.mod';}

\item \textbf{Optional: }Assign a value for the optional dummy variable
\texttt{run} if you decide whether or not you want to directly execute the
generated dynare. Assign \texttt{run=1;}

\item \textbf{Constant in state vector: }If there is a constant in the state vector $z$, you have to set \texttt{constant=1}. The constant has to be the first element of the $z$ vector.

\item \textbf{Calling the file:} Call the Hansen Sargent converter \ by
simply writing the following statement \texttt{HS\_Robust\_dyn};\newpage
\end{enumerate}

\subsection{How does the code work?}

\textbf{How does the code do it?} \bigskip The code does something
simple. It reads the matrixes from HS-LQ setup and declares variables and
equations in a mod.file using dynare syntax. \ The steps followed are:

\begin{enumerate}
\item Reading Dimensions of Matrixes

\begin{itemize}
\item For example, for the matrix \texttt{Phi\_c, }it uses the following
statement to set in memory the number of rows and columns:

[rr\_Phi\_c cc\_Phi\_c] =size(Phi\_c) ;
\end{itemize}

\item Checking Consistency:

\begin{itemize}
\item The matrixes must be conformable in HS-LQ setup. The code checks for
consistency along columns and rows:

\small
\begin{verbatim}

if  cc_Phi_c~=cc_Theta_h || cc_Pi~=cc_Theta_h || cc_Phi_i~=cc_Theta_k || ...
cc_Delta_h~=cc_Lambda || cc_Delta_k~=cc_Gamma || cc_U_b~=cc_U_d || cc_U_b~=cc_A22
disp('Error -- dimensions of matrices do not match... check number of columns')
end

if  rr_Phi_c~=rr_Phi_g || rr_Phi_c~=rr_Phi_g || rr_Phi_c~=rr_Gamma || ...
rr_Delta_k~=rr_Theta_k || rr_Delta_h~=rr_Theta_h || rr_Lambda~=rr_Pi || rr_A22~=rr_C2
disp('Error -- dimensions of matrices do not match... check number of rows')
end

\end{verbatim}
\normalsize
\texttt{\qquad disp('Error -- dimensions of matrices do not match... check
number of columns')}

\texttt{end}
\end{itemize}

\item Creating an m-file:

\begin{itemize}
\item This statement generates an object in memory that is used to record
what we are writing:

\texttt{fid = fopen(filename,'wt');}
\end{itemize}

\item Declaring Variables:

\begin{itemize}
\item For each of the matrixes in the LQ setup. The converter, will read the
number of columns and assign a variable. The number of columns corresponds
to the number of consumption variables. For example, it uses the following
statement to declare the consumption goods variables in the resource
constraint block:

\texttt{fprintf(fid,'\textbackslash n var ');}

\texttt{for i=1:(cc\_Phi\_c)}

\texttt{\qquad fprintf(fid,'c\%g ',i);}

\texttt{end}
\end{itemize}

\item Declaring Parameters:

\begin{itemize}
\item A similar proceduer is used to declare parameters.\ This block writes
up the paramter declaration followed by the variables \texttt{beta, theta, }%
and the elements in\texttt{\ Phi\_c}:

\texttt{fprintf(fid,'parameters ');}

\texttt{fprintf(fid,'beta ') ;}

\texttt{fprintf(fid,'theta ') ;}

\texttt{for j=1:rr\_Phi\_c}

\qquad \texttt{for i=1:cc\_Phi\_c}

\texttt{\ \qquad \qquad fprintf(fid,'Phi\_c\_\%g\%g ',j,i);}

\qquad \texttt{end}

\texttt{end}
\end{itemize}

\item Setting Parameter Values:

\begin{itemize}
\item Using a similar statement we declare parameters. The variable \texttt{%
\%g} in the statement is assign the value of the variable after the second comma. \texttt{\textbackslash n }means that the lines ends
there:

\texttt{fprintf(fid,'beta=\%g ;\textbackslash n',beta) ;}

\texttt{fprintf(fid,'theta=\%g ;\textbackslash n',theta) ;}

\texttt{for j=1:rr\_Phi\_c}

\texttt{\qquad for i=1:cc\_Phi\_c}

\texttt{\qquad \qquad fprintf(fid,'Phi\_c\_\%g\%g =\%g ;\textbackslash n',j,i,Phi\_c(j,i));}

\texttt{\qquad end}

\texttt{end}
\end{itemize}

\item Declaring Equations:

\begin{itemize}
\item This section of the code writes up the corresponding equation for the
system that characterizes the solution to the robust problem $\left( \ref%
{SYSTEM1}\right) .$ For example, to write in the first block of equations in
the system: ${\scriptsize M}_{t}^{d\prime }\Phi _{c}+{\scriptsize M}%
_{t}^{h\prime }\Theta _{h}=0$ we use the following commands:

\texttt{for i=1:(rr\_Phi\_c)}

\qquad \texttt{for j=1:(cc\_Phi\_c)}

\qquad \qquad \texttt{fprintf(fid,'Phi\_c\_\%g\%g*c\%g + ',i,j,j); }

\qquad \texttt{end}

\qquad \texttt{for j=1:(cc\_Phi\_g)}

\qquad \qquad \texttt{fprintf(fid,'Phi\_g\_\%g\%g*g\%g + ',i,j,j); }

\qquad \texttt{end}

\qquad \texttt{for j=1:(cc\_Phi\_i)}

\qquad \qquad \texttt{fprintf(fid,'Phi\_i\_\%g\%g*i\%g = ',i,j,j); }

\qquad \texttt{end}

\qquad \texttt{for j=1:(cc\_Gamma)}

\qquad \qquad \texttt{fprintf(fid,'Gamma\_\%g\%g*k\%g + ',i,j,j); }

\qquad \texttt{end}

\qquad \texttt{fprintf(fid,'d\%g',i); }

\qquad \texttt{fprintf(fid,'; \textbackslash n'); }

\texttt{end}
\end{itemize}

\item Running Dynare:

\begin{itemize}
\item This step simply follows the dynare syntax to call dynare according to the previously defined preferences:

\texttt{fprintf(fid,'shocks; \textbackslash n');}

\texttt{for i=1:cc\_C2}

\qquad \texttt{fprintf(fid,'var e\%g; \textbackslash n',i);}

\qquad \texttt{fprintf(fid,'stderr 1;  \textbackslash n');}

\texttt{end}

\texttt{fprintf(fid,'end; \textbackslash n )}%

\texttt{fprintf(fid,' \textbackslash n');}

\texttt{fprintf(fid,'steady; \textbackslash n \%
n');}

\begin{verbatim}fprintf(fid,'stoch\_simul(solve\_algo=3, periods=\%g); \n',irfperiods,periods); \end{verbatim}

\texttt{fclose(fid);}

\texttt{if (run)}

\qquad \texttt{dynare(filename)}

\texttt{end}
\end{itemize}
\end{enumerate}

\section{One Example}

This section shows how the model in Hansen, Sargent \& Tallarini can be solved using our code. Details on how to map a LQ decision problem to the form used here can be found in the LQ Hansen and Sargent Manuscript.


\subsection{Hansen, Sargent and Tallarini (1999, RES)}

\subsubsection{The Problem}

This is a version of the model of Hansen, Sargent and Tallarini (1999). The
planner solves the following problem:%
\begin{equation*}
\min_{\left\{ w_{t+1}\right\} }\max_{\left\{ c_{t},k_{t}\right\}
}E\{\sum_{t=0}^{\infty }\beta ^{t}-\left( 1/2\right) \left[
(s_{t}-b_{t})^{2}-\beta \Theta \frac{w_{t+1}\cdot w_{t+1}}{2}\right]
\end{equation*}%
The input of the utility function are an indirect service for the household, $%
s_{t},$ and\ a preference shock $b_{t}$ that satisfy the following:

\begin{equation*}
s_{t}=\Lambda h_{t-1}+\Pi c_{t}
\end{equation*}%
The resource constraint for this economy is a linear production function and

\begin{equation*}
c_{t}+i_{t}=\left( 1+r\right) k_{t-1}+y_{t}
\end{equation*}%
an exogenous productivity shift:

\begin{equation*}
y_{t}=U_{y}z_{t}
\end{equation*}

The Robust planner problem is different from the standard model in that: a
minimizer or "evil" agent chooses $w_{t+1},$ that will affect:%
\begin{equation*}
z_{t+1}=A_{z}z_{t}+C_{Z}w_{t+1}
\end{equation*}%
and we parameterize this to have a stable solution:

\begin{equation*}
\gamma :=\left( 1+r\right) =\frac{1}{\beta }
\end{equation*}%
For a pair of AR(2) processes, we have the following values of $A_{z}$ and $%
C_{Z}$:

$A_{z}=\left[
\begin{array}{ccccc}
1 & 0 & 0 & 0 & 0 \\
0 & \rho _{22} & \rho _{23} & 0 & 0 \\
0 & 1 & 0 & 0 & 0 \\
0 & 0 & 0 & \rho _{44} & \rho _{45} \\
0 & 0 & 0 & 1 & 0%
\end{array}%
\right] $ and $C_{z}=\left[
\begin{array}{cc}
0 & 0 \\
1 & 0 \\
0 & 0 \\
0 & 1 \\
0 & 0%
\end{array}%
\right] $ \\ $U_{y}=[5$ $1$ $0$ $1$ $0].$



\subsubsection{The Code}

The following code (\begin{verbatim} Example_HST.m \end{verbatim}) computes the equilibrium of the model described above.

\texttt{\% Declaration of Parameters}

\texttt{delta=0.05;}

\texttt{beta=1/1.05;}

\texttt{theta=10;}

\bigskip

\texttt{\% Matrix Form}

\texttt{Phi\_c=1 ; }

\texttt{Phi\_g=0 ;}

\texttt{Phi\_i=1 ;}

\texttt{Gamma=0.1 ;}

\texttt{Delta\_k=1-delta ;}

\texttt{Theta\_k=1 ; }

\texttt{Delta\_h=0 ; }

\texttt{Theta\_h=1 ;}

\texttt{Lambda=0 ;}

\texttt{Pi=1 ; }

\texttt{A22=[1 0 0 0 0; 0 0.1 0.2 0 0; 0 1 0 0 0; 0 0 0 0.1 0.2; 0 0 0 0 1];}

\texttt{C2=[0 0; 1 0; 0 0; 0 1; 0 0] ;}

\texttt{U\_b=[30 0 0 0 0] ;}

\texttt{U\_d=[5 1 0 1 0] ;}

\bigskip

\texttt{\%\% User Preferences for Dynare}

\texttt{periods=5000; \%parameters for simulation }

\texttt{irfperiods=15; \%parameters for irf's}

\texttt{filename='\texttt{HST\_Test\_AR2.mod}'; \%pick filename for mod file}

\texttt{\%Optional: decide whether or not you want to directly execute the
generated dynare}

\texttt{\%code (set run=1 to execute)}

\bigskip \texttt{run=1;}

\bigskip \texttt{constant=1;}

\texttt{\%set=1 if first element of z vector is a constant}

\texttt{\%\% Call the Hansen Sargent converter:}

\texttt{HS\_Robust\_dyn;}

\bigskip

This piece of matlab code generates a Mod-File that dynare is able to run. The matlab file is saved as \texttt{'\texttt{HST\_Test\_AR2.mod}'.
} We can alter this mod file to
be able to estimate the model.

%\subsubsection{Results}
%
%Here we plot the sensibility to the parameter $\theta $ in Hansen, Sargent
%and Tallarini's Model. The sensibility is done over the Impulse Response.
%The first figure plots the impulse to both, transitory and permanent shocks,
%of consumption. Dotted lines represent a stronger concern for Robustness.
%
%\FRAME{ftbpFU}{7.5905in}{4.1122in}{0pt}{\Qcb{Response of Consumption}}{}{%
%consumptionhst.tif}{\special{language "Scientific Word";type
%"GRAPHIC";maintain-aspect-ratio TRUE;display "USEDEF";valid_file "F";width
%7.5905in;height 4.1122in;depth 0pt;original-width 13.3017in;original-height
%7.1875in;cropleft "0.03305";croptop "0.994903";cropright
%"1.03305";cropbottom "-0.005097";filename 'Boston FED
%Presentation/ConsumptionHST.tif';file-properties "XNPEU";}}
%
%This figure plots the impulse response of investment again to both shocks.
%
%\FRAME{ftbpFU}{7.3198in}{3.9652in}{0pt}{\Qcb{Response of Investment}}{}{%
%investmenthst.tif}{\special{language "Scientific Word";type
%"GRAPHIC";maintain-aspect-ratio TRUE;display "USEDEF";valid_file "F";width
%7.3198in;height 3.9652in;depth 0pt;original-width 13.3017in;original-height
%7.1875in;cropleft "0.029999";croptop "1";cropright "1.029999";cropbottom
%"0";filename 'Boston FED Presentation/InvestmentHST.tif';file-properties
%"XNPEU";}}
%
%%\section{Other Examples}


\end{document}
